% Homework template for Inference and Information
% UPDATE: September 26, 2017 by Xiangxiang
\documentclass[a4paper]{article}
\usepackage{amsmath, amssymb, amsthm}
% amsmath: equation*, amssymb: mathbb, amsthm: proof
\usepackage{moreenum}
\usepackage{mathtools}
\usepackage{url}
\usepackage{enumitem}
\usepackage{graphicx}
\usepackage{subcaption}
\usepackage{booktabs} % toprule
\usepackage[mathcal]{eucal}
\input{iidef}
\begin{document}
\courseheader

\newcounter{hwcnt}
\setcounter{hwcnt}{10} % set to the times of Homework

\begin{center}
  \underline{第\thehwcnt 周作业} \\
\end{center}
\begin{flushleft}
  赵丰\quad \studentID\hfill
  \today
\end{flushleft}
\hrule

\vspace{2em}

\flushleft
\rule{\textwidth}{1pt}
\begin{itemize}
\item {\bf Acknowledgments: \/} 
  This coursework refers to textbook.  
\item {\bf Collaborators: \/}
  \begin{itemize}
  \item 22 was solved with the help from solution manual of the textbook.
  \item 27 was solved with the help from solution manual of the textbook.  
  \end{itemize}
\end{itemize}
\rule{\textwidth}{1pt}

\vspace{2em}

I use \texttt{enumerate} to generate answers for each question:

\begin{enumerate}[label=\arabic*.]
  \setlength{\itemsep}{3\parskip}
  \setcounter{enumi}{2}
  \item 
     首先“有效招聘问题”是属于\NP 的,因为至多$k$位申请人可以作为验证算法的有效证书,使得在多项式时间里可以判定
     对给定的$m$位申请人原问题有解。
     
     其次我们将点覆盖(Vertex Cover)问题多项式归约到有效招聘问题。对给定的图$G(V,E)$,将$V$集合作为申请人的集合,$|G|=m$;
     $E$集合作为运动的集合,$|E|=n$;
     若$v_1,v_2$有边$e$相连,则存在申请人集合的子集$\{v_1,v_2\}$,使得集合中的两位申请人对$e$运动都有资格。
     因此
     图$G$中存在点集$S\subset V$,使得$|S|\leq k$,且$S$是$G$的点覆盖集,当且仅当
     可以至多聘用$k$个管理人员并且对这$n$项运动中
     的每一项至少有一位管理人员有资格。
     从而点覆盖问题可以多项式归约到有效招聘问题。
     因为点覆盖问题是\NP 完全的,从而有效招聘问题也是\NP 完全的。
  \setcounter{enumi}{21}
  
    \begin{algorithm}
    \caption{调用$\mathcal{A}$,在多项式时间内解决独立集问题}\label{alg}
    \begin{algorithmic}[1]
    \STATE 通过扩展的BFS算法,计算$G$的连通分支的个数$t(\leq |V|)$
    \IF{$t==1$}
    \RETURN 调用算法$\mathcal{A}(G,k)$
    \ENDIF
    
    \STATE $m=0$
    \FOR{$G$的每一连通分支$G_i=(V_i,E_i)$}
    \FOR{$k_i=2$ to $|V_i|$}
    \STATE $t_i \leftarrow $ 调用算法$\mathcal{A}(G_i,k_i)$
    \IF{$t_i$ 为假}
    \STATE \textbf{break}
    \ENDIF
    \ENDFOR
    \STATE $m=m+k_i-1$
    \ENDFOR
    \IF{$m\geq k$}
    \RETURN yes
    \ELSE 
    \RETURN no
    \ENDIF
    \end{algorithmic}
    \end{algorithm}  
     % 我们知道通过BFS算法可以在$G$的大小的多项式时间内判断$G$是否是连通的,如果$G$是连通的,那么直接
     % 调用算法$\mathcal{A}$即可给出独立集的判断。如果$G$不是连通的,可以能过扩展BFS算法计算出$G$的连通分支的个数$(\leq |V|)$。
     % 对每一连通分支$G_i=(V_i,E_i)$,从$k=1(\mathcal{A})$返回yes 开始,一直计算到$\
     % Problem...  
    \item (也可以考虑构造一个新的图,取一个新的顶点与$G$的所有顶点均相连,则对于$k\geq 2$的独立集一定不包含这个顶点,
    但此时新的图$G'$是连通的,只需对$G'$运行一次$\mathcal{A}$算法即可对图$G$是否有大小不小于$k$的独立集做出判断)
    表\ref{alg}给出的算法调用算法$\mathcal{A}$(多项式时间算法)的次数被$|V|^2$控制住,因此表\ref{alg}给出的算法也是多项式时间算法。
    
  \setcounter{enumi}{6}
    \item 容易说明 存在$n$个彼此没有公共元素的4元组当且仅当有$n$个元组形成对$W\cup X \cup Y \cup Z $的一个集合覆盖。
    首先可证明4维匹配是属于\NP 的,只需使用$n$个4元组作为验证证书,便可以在多项式时间内判定这$n$个4元组是否形成了对
    $W\cup X \cup Y \cup Z $的覆盖。
    其次我们把3维匹配多项式归约到4维匹配。给定集合$X,Y,Z$,大小均为$n$,以及三元组的集合$T\subset X \times Y \times Z$
    任给大小为$n$的集合$W$满足与$X\cup Y\cup Z$不交,构造4元组的集合$T'=\{(w,x,y,z)|(x,y,z)\in T,w\in W\}$。
    
    若原3维匹配问题存在覆盖
    $X\cup Y \cup Z \subset \cup_{i=1}^n\{x_i,y_i,z_i\}$,取$(w_i,x_i,y_i,z_i)\in T'$,则
    $W \cup X\cup Y \cup Z \subset \cup_{i=1}^n\{w_i,x_i,y_i,z_i\}$,即构造出的4维匹配问题存在覆盖。

    若构造出的4维匹配问题存在覆盖
    $W \cup X\cup Y \cup Z \subset \cup_{i=1}^n\{w_i,x_i,y_i,z_i\}$,因为$W$与$X\cup Y\cup Z$不交,$|W|=n$,
    则有$W=\cup_{i=1}^n\{w_i\}$。又因为$\{w_i,x_i,y_i,z_i\}\in T'$,所以$\{x_i,y_i,z_i\}\in T$,
    因此取$(x_i,y_i,z_i)\in T$,
    则$X\cup Y \cup Z \subset \cup_{i=1}^n\{x_i,y_i,z_i\}$。    
    即原3维匹配问题存在覆盖。

    从而3维匹配可以多项式归约到4维匹配,即4维匹配是\NP 完全的。
    
  \setcounter{enumi}{26}
    \item 
    首先可证明原问题是属于\NP 的,只需使用$k$个子集合$S_k$作为验证证书,便可以在多项式时间内判定所给不等式是否成立。
    
    其次我们把数的划分问题多项式归约到原问题。
    给定正数$x_1,\dots,x_n$,记$s=\sum_{i=1}^n x_i$,
    取$k=2,B=\frac{s^2}{2}$,若数的划分问题存在$S_1,S_2$不交,且
    $\sum_{x_i\in S_1}x_i=\sum_{x_i\in S_2}x_i=\frac{s}{2}$,则$S_1,S_2$满足
    \begin{equation}
        (\sum_{x_i\in S_1}x_i)^2+(\sum_{x_i\in S_2}x_i)^2=\frac{s^2}{2}=B
    \end{equation}
    即$S_1,S_2$是原问题的解。
    
    反之,假设存在$S_1,S_2$,记$s_1=\sum_{x_i\in S_1}x_i,s_2=\sum_{x_i\in S_2}x_i$,使得
    $s_1^2+s_2^2\leq \frac{s^2}{2}$,又$s_1+s_2=s$。由算术平均值小于平方平均值:
    \begin{equation}\label{eq:a2}
        \left(\frac{s_1+s_2}{2}\right)\leq \frac{s_1^2+s_2^2}{2}
    \end{equation}
    于是得$s_1^2+s_2^2\geq \frac{s^2}{2}$,故$s_1^2+s_2^2= \frac{s^2}{2}$,且式\eqref{eq:a2}等号成立当且仅当$s_1=s_2$,从而
    $S_1,S_2$是数的划分问题的解。
    
    数的划分问题可以多项式归约到原问题,即原问题是\NP 完全的。
    
  \end{enumerate}

\end{document}
%%% Local Variables:
%%% mode: latex
%%% TeX-master: t
%%% End:
 