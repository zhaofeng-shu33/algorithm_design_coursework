% Homework template for Inference and Information
% UPDATE: September 26, 2017 by Xiangxiang
\documentclass[a4paper]{article}
\usepackage{amsmath, amssymb, amsthm}
% amsmath: equation*, amssymb: mathbb, amsthm: proof
\usepackage{moreenum}
\usepackage{mathtools}
\usepackage{url}
\usepackage{enumitem}
\usepackage{graphicx}
\usepackage{subcaption}
\usepackage{booktabs} % toprule
\usepackage[mathcal]{eucal}
\DeclarePairedDelimiter{\ceil}{\lceil}{\rceil}

%% Definitions for Inference and Information
%% UPDATE: September 26, 2017 by Xiangxiang 
\newcommand{\theterm}{Fall 2017}
\usepackage{xeCJK}

\setCJKmainfont[AutoFakeBold]{SimSun}

\newcommand{\thecoursename}{
Tsinghua University\\
%\vspace*{0.1in}
算法设计与分析
}
\newcommand{\studentID}{2017310711}
\newcommand{\courseheader}{
\vspace*{-1in}
\begin{center}
\thecoursename \\
\theterm
\vspace*{0.1in}
\hrule
\end{center}
}
\newcommand{\NP}{$\mathcal{NP}$}
\newcommand{\rvx}{\mathsf{x}}    % x, r.v.
\newcommand{\rvy}{\mathsf{y}}    % y, r.v.
\newcommand{\rvz}{\mathsf{z}}    % z, r.v.
\newcommand{\urvx}{\underline{\mathsf{x}}}    % x, r.v. vec
\newcommand{\urvy}{\underline{\mathsf{y}}}    % y, r.v. vec
\newcommand{\urvz}{\underline{\mathsf{z}}}    % z, r.v. vec
\newcommand{\defas}{\triangleq} %\coloneqq
\newcommand{\reals}{\mathbb{R}}
%\newcommand{\T}{\mathrm{T}}    % transpose

% \newcommand{\E}[1]{\mathbb{E}\left[{#1}\right]}
% \newcommand{\Prob}[1]{\mathbb{P}\left({#1}\right)}
\DeclareMathOperator*{\argmax}{arg\,max}
\DeclareMathOperator*{\argmin}{arg\,min}
\DeclareMathOperator*{\argsup}{arg\,sup}
\DeclareMathOperator*{\arginf}{arg\,inf}
\DeclareMathOperator{\Var}{Var}
\DeclareMathOperator{\Cov}{Cov}
\DeclareMathOperator{\MSE}{MSE}
\DeclareMathOperator{\In}{\mathbb{I}}
\DeclareMathOperator{\E}{\mathbb{E}}
\DeclareMathOperator{\Prob}{\mathbb{P}}
\DeclareMathOperator{\sign}{sign}
\DeclareMathOperator{\st}{s.t.}
\newcommand\independent{\protect\mathpalette{\protect\independenT}{\perp}}
\def\independenT#1#2{\mathrel{\rlap{$#1#2$}\mkern2mu{#1#2}}}
\usepackage{algorithmic}
\usepackage{algorithm}

\newtheorem{Def}{Definition}
\begin{document}
\courseheader

\newcounter{hwcnt}
\setcounter{hwcnt}{13} % set to the times of Homework

\begin{center}
  \underline{第\thehwcnt 周作业} \\
\end{center}
\begin{flushleft}
  赵丰\quad \studentID\hfill
  \today
\end{flushleft}
\hrule

\vspace{2em}

\flushleft
\rule{\textwidth}{1pt}
\begin{itemize}
\item {\bf Acknowledgments: \/} 
  This coursework refers to textbook.  
\item {\bf Collaborators: \/}
  \begin{itemize}
  \item 3 was solved with the help from solution manual of the textbook.    
  \end{itemize}
\end{itemize}
\rule{\textwidth}{1pt}

\vspace{2em}

I use \texttt{enumerate} to generate answers for each question:

\begin{enumerate}[label=\arabic*.]
  \setlength{\itemsep}{3\parskip}
  \setcounter{enumi}{1}
  \item 
    \begin{enumerate}[label=(\alph*)]
    \item 考虑$V=\{1,2,3,4\},E=\{(1,3),(1,4),(2,3)\},G=(V,E)$,如果梯度上升法第一次选取了边$(1,3)$,则最终返回的
    匹配数为1,而实际上$G$的最大匹配数为2。
    \item 匹配$M$中共有$2|M|$个节点,记为$V'$,反设$\forall e'\in M',M\cup \{e'\}$都不是$G$的匹配,那么$e'$有一个节点属于$V'$,
    因为$|M'|>|V'|$,由鸽笼原理,至少存在两条边$e'_1\in M',e'_2 \in M'$有公共的顶点,这与$M'$是$G$的匹配矛盾。
    因此,存在一条边$e'\in M',$使得$M\cup \{e'\}$是$G$的匹配。
    \item 设$M'$是(b)中的最优匹配,$M$是梯度上升法返回的匹配,由梯度上升法的性质,若$|M'|>2|M|$,
    则$G$中存在可以加入$M$的边,算法不会终止。因此有$\frac{1}{2}|M'|\leq |M|$,即梯度上升法返回的局部最优匹配
    至少有$G$的最大匹配的一半大。

    \end{enumerate}
    
    \item \begin{enumerate}[label=(\alph*)]
    \item 由对称性不妨设$T_1\geq T_2$,只需证$T_1 \leq 2T_2$,
    由$\forall j,t_j\leq \frac{1}{2}\sum_{i=1}^n t_i$可知$M_1$至少分到了2个作业,设$t_j$是$M_1$用时最短的作业,则
    $T_1\geq 2t_j$。由局部搜索算法的性质,将作业$j$从$A$移到$B$会导致两台机器处理时间差的绝对值增大,即
    $(T_2+t_j)-(T_1-t_j)\geq T_1-T_2$,整理得$T_1\leq T_2+t_j$,若$T_2<t_j$,推出$T_1\leq 2t_j$矛盾。因
    此$t_j\leq T_2$,从而推出$T_1\leq 2T_2$。
    \item 首先我们给出,若作业$j$有性质$t_j\geq |T_1-T_2|$,其中$T_1,T_2$是当前两机器分配的总作业时长,
    则作业$j$不会被移动。反设不然,且$j\in A,T_1\geq T_2$,
    若$T_1-t_j\geq T_2+t_j$,则 $2t_j\leq T_1-T_2\leq t_j$,推出$t_j=0$矛盾。因此$T_1-t_j\leq T_2+t_j$
    由局部搜索算法的性质,$(T_2+t_j)-(T_1-t_j)<T_1-T_2$,从而有$t_j<T_1-T_2$与$t_j\geq T_1-T_2$矛盾。
    根据(a)中已有的推导,在$T_1\geq T_2$的情况下,设$t_j=\max_{i\in A}t_i$,
    并设连续的把$A$中作业搬到$B$中的最后一步搬运的作业时长为$t_k$,则由每次搬运时间最大的性质,$t_j\geq t_k$,
    设搬运后两台机器时长为$T_1',T_2'$,则满足$T_1'\leq T_2'$,而搬运前满足$T_1'+t_k >T_2'-t_k$,根据绝对值减小的规则
    可推出$T_2'-T_1'< (T_1'+t_k)-(T_2'-t_k)$,即$t_k > T_2'-T_1'$。
    又$T_2'>T_1'$,显然有$t_k>T_1'-T_2'$,所以$t_k>|T_2'-T_1'|$
    从而$t_j\geq |T_2'-T_1'|$。
    由于每次搬运时间差会减小,对之后的步骤$t_j\geq |T_2-T_1|$总是成立。由一开始的结论,作业$j$不会再次被搬走。
    因此每个作业最多搬运一次。
    \item 设四个作业用时分别为$5,5,7,7$,满足(a)中的没有任一作业占绝对优势的条件,一开始所有作业均分给$A$,按照(b)给出的
    最大时长的局部搜索算法,最终返回的结果是$A=\{5,5\},B=\{7,7\}$,时差为4。而显然最优解时差为0。
    \end{enumerate}
  \item 
    \begin{enumerate}[label=(\alph*)]
    \item (势函数)认为一步是一次交换,设第$i$次交换后指标$L(i)=\sum_{j\neq k} \max(T_j(i),T_k(i))$,
    其中$T_j(i)$是第$i$次交换后第$j$台机器上的负载。
    %所有机器上的最大负载在第$i$次交换后满足不等式关系$M(i)=\max_j T_j(i)\leq L(i)$。
    根据启发式局部搜索的性质$L(i)$是关于$i$的严格递减函数,因此对给定的
    初始作业分配方案,至多经过$L(0)$步$L(i)$减为零。而$L(i)$具有非负性,从而得到算法至多$L(0)$步终止。由于算法终止的条件是
    得到一个稳定的分配,从而得证。
    \item 设最后得到的稳定分配第$i$台机器作业总时间最长,其最后一个分配的作业时长为$t_j$,由算法的性质,如果把作业$j$分给第$k$台机器,
    则有$\max\{L_k,L_i\}\leq \max\{L_k+t_j,L_i-t_j\}$,因为$L_i\geq L_k$,只能有$L_i\leq L_k+t_j$。这个不等式关系对$1\leq k\leq m$成立。
    与11.1节有完全相同的推导,利用最短工期的两个性质$L^*\geq \max_j t_j$和$L^*\geq \frac{1}{m}\sum_j t_j$可以证明$L_i\leq 2L^*$。
    即任何稳定分配的工期不超过最短工期的两倍。
    \end{enumerate}
    
  \end{enumerate}

\end{document}
%%% Local Variables:
%%% mode: latex
%%% TeX-master: t
%%% End:
    \begin{equation}
    \end{equation}
