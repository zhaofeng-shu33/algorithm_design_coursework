\documentclass{article}
\usepackage{xeCJK}
\setCJKmainfont[AutoFakeBold]{SimSun}
\usepackage{amsmath,esint,amssymb,amsthm}
\usepackage{algorithmic}
\usepackage{algorithm}
\usepackage{mathtools}

\DeclarePairedDelimiter{\floor}{\lfloor}{\rfloor}
\begin{document}
\title{第六周作业}
\author{赵丰,2017310711}
\maketitle

\textbf{Ch5.1}
\begin{algorithm}
\caption{$O(\log n)$次查找中位数的算法}
\begin{algorithmic}[1]
\STATE 初始化两个数据库的查找范围为$r_1=[0,n),r_2=[0,n)$
\STATE 初始化两个数据库的查找元素指针$p_1,p_2$均为$\floor{\frac{n}{2}}$,即查询每个数据库的中位数
\WHILE{$r_1$和$r_2$均非空}
\IF{$d(p_1)<d(p_2)$}
\STATE $r_1 \leftarrow [\frac{r_1(f)+r_1(b)}{2},r_1(b))$
\STATE $r_2 \leftarrow [r_2(a),\frac{r_2(f)+r_2(b)}{2})$
\ELSE 
\STATE $r_2 \leftarrow [\frac{r_2(f)+r_2(b)}{2},r_2(b))$
\STATE $r_1 \leftarrow [r_1(a),\frac{r_1(f)+r_1(b)}{2})$
\ENDIF
\STATE $p_1 \leftarrow \floor{\frac{r_1(f)+r_1(b)}{2}}$
\STATE $p_2 \leftarrow \floor{\frac{r_2(f)+r_2(b)}{2}}$
\ENDWHILE
\RETURN $d(p_1)$和$d(p_2)$较小的数作为$2n$个数据的中位数
\end{algorithmic}
\end{algorithm}

上面算法中$[i,j)$表示从$i到j$的整数但不包括$j$,$d(p_1)$表示$p_1$指向的数据库1中的数据,
若$r_2=[i,j)$,则$r_2(f)$表示$i$,$r_2(b)$表示$j$,$\floor{x}$表示对浮点数$x$向下取整。

该算法利用二分查找的方法以$O(\log n)$的时间复杂度找到了$2n$个数据的中位数。

\textbf{Ch5.2}
类似计算逆序数的算法给出计算一个序列重要的逆序的算法如表\ref{alm:ch52}所示

\begin{algorithm}
\caption{sort\_and\_count(L)}\label{alm:ch52}
\begin{algorithmic}[1]
\IF{$L$长度为1}
\RETURN 0
\ENDIF
\STATE $r_A \leftarrow $ sort\_and\_count(A)
\STATE $r_B \leftarrow $ sort\_and\_count(B)
\STATE 初始化数组$C$,长度与$L$相同,初始化指针$p_A=0,p_B=0,r_C=0$
\FOR {$p$ 从 $1$ 到 $C$的最后一个元素}
\IF{$p_B=$length(B) 或者 $A(p_A)\leq 2 B(p_B)$}
\STATE $C(p)\leftarrow A(p_A)$
\STATE $p_A \leftarrow p_A+1$
\ELSE
\STATE $r_C \leftarrow r_C+$length(A)$-p_A$
\STATE $C(p)\leftarrow B(p_B)$
\STATE $p_B \leftarrow p_B+1$
\ENDIF
\ENDFOR
\STATE $L \leftarrow C$ 
% L is revised!
\RETURN $r_A+r_B+r_C$
\end{algorithmic}
\end{algorithm}


\textbf{Master定理}

设$T(n),f(n)$是一个以正整数为自变量的增函数,且满足递推关系
\begin{equation}\label{eq:DefTn}
T(n)=aT(\floor{\frac{n}{b}})+f(n)
\end{equation}
其中$a,b$是大于1的正整数,设$c=\log_b a$,那么可以分三种情形讨论$T(n)$的增长阶:
\begin{enumerate}
\item 若$f(n)=O(n^{c-\epsilon})$,其中$\epsilon>0$,那么$T(n)=\Theta(n^c)$
\item 若$f(n)=\Theta(n^c \log^k n),\forall k\geq 0$,那么$T(n)=\Theta(n^c \log^{k+1} n)$
\item 若$f(n)=\Omega(n^{c+\epsilon})$,其中$\epsilon>0$,并且存在常数$q<1$,使得对于充分大的$n$,有
$af(\frac{n}{b})\leq qf(n)$,那么$T(n)=\Theta (f(n))$
\end{enumerate}

\begin{proof}

对于情形1,我们令$n=b^x$,并设$m=x-\floor{x}$,
于是$T(n)=T(b^x)$,并且我们取$T(b^m)$为一个已知的运行时间。
令$G(x):=T(b^x)$,所以我们有
\begin{equation}
G(x)=aG(x-1)+f(b^x)
\end{equation}
递推得 
\begin{equation}\label{eq:Rec}
G(x)=a^{\floor{x}}  G(m)+ \sum_{i=0}^{\floor{x}-1}a^i f(b^{\floor{x}-i+m})
\end{equation}
根据已知条件,存在常数$C$,使得$f(n)\leq C n^{c-\epsilon}$,所以由等比数列求和公式,上式
可放缩为:
\begin{equation}\label{eq:Gxm}
G(x)\leq a^{\floor{x}}  G(m)+ C \sum_{i=0}^{\floor{x}-1}a^i b^{(\floor{x}-i+m)(c-\epsilon)} 
\end{equation}
记$b_1=b^{c-\epsilon}$,上式化为
\begin{equation}
G(x)\leq a^{\floor{x}}  G(m)+C b_1^m \frac{a^{\floor{x}}-b_1^{\floor{x}}}{\frac{a}{b_1}-1}
\end{equation}

我们来比较一下$b_1$与$a$的大小从而确定$G(x)$被控制的量阶,
因为
\begin{equation}
\log b_1 = (c-\epsilon)\log b =\log a - \epsilon \log b < \log a
\end{equation}
所以\eqref{eq:Gxm}式中$a^{\floor{x}}$为主要控制项
进一步放缩得:
\begin{equation}
a^{\floor{x}} \leq a^x = a^{\frac{\log n}{\log b} }= n^c
\end{equation}
因此$T(n)=O(n^c)$,另一方面由\eqref{eq:Rec}式可得
\begin{equation}
G(x)\geq a^{\floor{x}}  G(m) \geq C_1 a^x
\end{equation}
因此$T(n)=\Omega(n^c)$,所以$T(n)=\Theta(n^c)$

对于情形2,易知$a=b^c$,由\eqref{eq:Rec}式,可得
\begin{align*}
G(x) \sim & a^{\floor{x}}  G(m)+ \sum_{i=0}^{\floor{x}-1} a^i b^{c(\floor{x}-i+m)} ((\floor{x}-i+m)\log b)^k\\
= &  a^{\floor{x}}  G(m) + a^{\floor{x}+m} (\log b)^k\sum_{i=1}^{\floor{x}}(i+m)^k  \\
\sim & a^{\floor{x}}  G(m) + a^{\floor{x}+m} (\log b)^k\int_{1}^{\floor{x}}(y+m)^kdy \\
\sim & C_1 a^{\floor{x}}(\floor{x}+m)^{k+1} \\
\sim & n^c (\log n)^{k+1}
\end{align*}

对于情形3,由\eqref{eq:Rec}式以及$af(b^{x-1})\leq qf(b^x),\forall x>N$
有
\begin{align*}
G(x) \leq & a^{\floor{x}}  G(m)+ \sum_{i=0}^{N} a^i f(b^{\floor{x}-i+m})+ \sum_{i=N}^{\floor{x}-1}q^i f(b^{\floor{x}+m}), |q|<1\\
\leq  & a^{\floor{x}}  G(m)+ \frac{q^N}{1-q}f(b^{\floor{x}+m})+C_2\\
\sim & f(n),\text{by }f(n)=\Omega(n^{c+\epsilon})
\end{align*}
所以$T(n)=G(x)=O(f(n))$
另外由\eqref{eq:DefTn}式有$T(n)\geq f(n)$,所以$T(n)=\Omega(f(n))$
从而有$T(n)=\Theta(f(n))$。
\end{proof}
\end{document}
