% Homework template for Inference and Information
% UPDATE: September 26, 2017 by Xiangxiang
\documentclass[a4paper]{article}
\usepackage{amsmath, amssymb, amsthm}
% amsmath: equation*, amssymb: mathbb, amsthm: proof
\usepackage{moreenum}
\usepackage{mathtools}
\usepackage{url}
\usepackage{enumitem}
\usepackage{graphicx}
\usepackage{subcaption}
\usepackage{booktabs} % toprule
\usepackage[mathcal]{eucal}
%% Definitions for Inference and Information
%% UPDATE: September 26, 2017 by Xiangxiang 
\newcommand{\theterm}{Fall 2017}
\usepackage{xeCJK}

\setCJKmainfont[AutoFakeBold]{SimSun}

\newcommand{\thecoursename}{
Tsinghua University\\
%\vspace*{0.1in}
算法设计与分析
}
\newcommand{\studentID}{2017310711}
\newcommand{\courseheader}{
\vspace*{-1in}
\begin{center}
\thecoursename \\
\theterm
\vspace*{0.1in}
\hrule
\end{center}
}
\newcommand{\NP}{$\mathcal{NP}$}
\newcommand{\rvx}{\mathsf{x}}    % x, r.v.
\newcommand{\rvy}{\mathsf{y}}    % y, r.v.
\newcommand{\rvz}{\mathsf{z}}    % z, r.v.
\newcommand{\urvx}{\underline{\mathsf{x}}}    % x, r.v. vec
\newcommand{\urvy}{\underline{\mathsf{y}}}    % y, r.v. vec
\newcommand{\urvz}{\underline{\mathsf{z}}}    % z, r.v. vec
\newcommand{\defas}{\triangleq} %\coloneqq
\newcommand{\reals}{\mathbb{R}}
%\newcommand{\T}{\mathrm{T}}    % transpose

% \newcommand{\E}[1]{\mathbb{E}\left[{#1}\right]}
% \newcommand{\Prob}[1]{\mathbb{P}\left({#1}\right)}
\DeclareMathOperator*{\argmax}{arg\,max}
\DeclareMathOperator*{\argmin}{arg\,min}
\DeclareMathOperator*{\argsup}{arg\,sup}
\DeclareMathOperator*{\arginf}{arg\,inf}
\DeclareMathOperator{\Var}{Var}
\DeclareMathOperator{\Cov}{Cov}
\DeclareMathOperator{\MSE}{MSE}
\DeclareMathOperator{\In}{\mathbb{I}}
\DeclareMathOperator{\E}{\mathbb{E}}
\DeclareMathOperator{\Prob}{\mathbb{P}}
\DeclareMathOperator{\sign}{sign}
\DeclareMathOperator{\st}{s.t.}
\newcommand\independent{\protect\mathpalette{\protect\independenT}{\perp}}
\def\independenT#1#2{\mathrel{\rlap{$#1#2$}\mkern2mu{#1#2}}}
\usepackage{algorithmic}
\usepackage{algorithm}

\begin{document}
\courseheader

\newcounter{hwcnt}
\setcounter{hwcnt}{9} % set to the times of Homework

\begin{center}
  \underline{第\thehwcnt 周作业} \\
\end{center}
\begin{flushleft}
  赵丰\quad \studentID\hfill
  \today
\end{flushleft}
\hrule

\vspace{2em}

\flushleft
\rule{\textwidth}{1pt}
\begin{itemize}
\item {\bf Acknowledgments: \/} 
  This coursework refers to textbook.  
\item {\bf Collaborators: \/}
  \begin{itemize}
  \item 23 was solved with the help from solution manual of the textbook.
  \end{itemize}
\end{itemize}
\rule{\textwidth}{1pt}

\vspace{2em}

I use \texttt{enumerate} to generate answers for each question:

\begin{enumerate}[label=\arabic*.]
  \setlength{\itemsep}{3\parskip}
  \setcounter{enumi}{5}
  \item 
    \begin{enumerate}[label=(\alph*)]
    \item 首先该问题在\NP 类中,因为对给定的$x_1,x_N$的真值赋值作为判定证书,可以在多项式时间内判断赋值中1的
    个数不超过$k$并且对各从句进行计算判断所有的从句是否均为1,满足这两条则实例属于该问题类。
    
    \item 对于给定的3-SAT 问题,我们用如下的方法将其归约为“少量真变量的单调可满足性问题”,
    首先在每一个从句中将含有negation的
    literal看作独立布尔变量,假如原3-SAT问题中变量中$x_i$取过反,
    则相应的引入新的从句$x_i\cup \bar{x}_i$。
    设3-SAT中总共取反的布尔变量数量为$t$,则我们在原来的己有从句基础上加了$t$个$x_i\cup \bar{x}_i$从句作为
    “少量真变量的单调可满足性问题”的输入。
    我们下面证明
    “3-SAT” 满足 当且仅当
    “少量真变量的单调可满足性问题”得到满足
    至多有$t$个变量置1。
    
    首先设
    3-SAT问题有一组布尔变量使所有从句为真,则对于“少量真变量的单调可满足性问题”,
    通过扩展$t$个$\bar{x}_i$的取值为$x_i$的否,可以得到恰有$t$个变量置1的一组满足性问题。
    
    
    反之,若存在至多有$t$个变量置1的一组满足性问题,
    注意到由于加入了从句$x_i\cup \bar{x}_i$,所以$x_i$与$\bar{x}_i$不可能同时为0,
    另一方面,由于$x_i,\bar{x}_i$中至少有1个取1,又总共取1的变量个数不超过$t$,所以
    $x_i$与$\bar{x}_i$不可能同时为1,这样就满足了negation对变量关系的约束,
    从而我们取诸$x_j$成为原3-SAT问题的判定为真的解。
    \item 由于3-SAT问题是属于NP-complete 类中的,3-SAT问题可以归约到题中问题,所以题中问题也是NP-complete的。
    \end{enumerate}
  \setcounter{enumi}{22}    
  \item 若$u$是$A$和$B$上的连接,
  % 从$u$的头开始,可找到$a_i,b_j$分别为$u$的从头超始的某个子串,若$a_i=b_j$,则$a_i$即为要找的
  % 长度被$A\cup B$总长多项式控制住的的$A\cup B$上的连接。否则不妨设$b_j$是$a_i$严格的从头超始的子串。
  % 则可找到最小的$q$,与$b_{j1},\dots,b_{jq}$,使得$a_i$是$b_{j}b_{j1}\dots b_{jq}$从头超始的子串,并且
  % $b_{j}b_{j1}\dots b_{jq}$是$u$从头超始的子串。
  % 这里我们只讨论严格的情形,在$a_i=b_{j}b_{j1}\dots b_{jq}$时结论是平凡的。
  % 注意到$b_{j}b_{j1}\dots b_{jq}$的长度被$a_i$与$b_{jq}$的长度之和控制住。  
  记$u$某位置$p$上字符$\in$ 类$A(a_i,k)$,当$a_i$是在$u$的该位置的子串且$u(p)=a_i(k)$。
  同理规定$p\in B(b_j,k)$的含义。$u$中每隔两类的总数目为$A$的长度和$L(A)+1$即会出现2个位置属于相同的$A$类。
  如果$u$的长度超过$L(B)+1$个$L(A)+1$,对于这$L(B)+1$对位置,它们在$A$中的类相同,至少有1对它们在$B$中的类相同。
  当把2个位置中间的元素去掉后,剩下的$u$仍是$A\cup B$上的连接。因此,我们从一个长度超过$L(B)+1$个$L(A)+1$的$u$出发,
  总可以得到一个长度不超过$L(B)+1$个$L(A)+1$的串。这个量被$(L(A+B)+1)^2$控制住,因此可作为一个串是否是$A\cup B$上的连接的判定证书。
\end{enumerate}

\end{document}
%%% Local Variables:
%%% mode: latex
%%% TeX-master: t
%%% End:
 