% Homework template for Inference and Information
% UPDATE: September 26, 2017 by Xiangxiang
\documentclass[a4paper]{article}
\usepackage{amsmath, amssymb, amsthm}
% amsmath: equation*, amssymb: mathbb, amsthm: proof
\usepackage{moreenum}
\usepackage{mathtools}
\usepackage{url}
\usepackage{enumitem}
\usepackage{graphicx}
\usepackage{subcaption}
\usepackage{booktabs} % toprule
\usepackage[mathcal]{eucal}
\DeclarePairedDelimiter{\ceil}{\lceil}{\rceil}

%% Definitions for Inference and Information
%% UPDATE: September 26, 2017 by Xiangxiang 
\newcommand{\theterm}{Fall 2017}
\usepackage{xeCJK}

\setCJKmainfont[AutoFakeBold]{SimSun}

\newcommand{\thecoursename}{
Tsinghua University\\
%\vspace*{0.1in}
算法设计与分析
}
\newcommand{\studentID}{2017310711}
\newcommand{\courseheader}{
\vspace*{-1in}
\begin{center}
\thecoursename \\
\theterm
\vspace*{0.1in}
\hrule
\end{center}
}
\newcommand{\NP}{$\mathcal{NP}$}
\newcommand{\rvx}{\mathsf{x}}    % x, r.v.
\newcommand{\rvy}{\mathsf{y}}    % y, r.v.
\newcommand{\rvz}{\mathsf{z}}    % z, r.v.
\newcommand{\urvx}{\underline{\mathsf{x}}}    % x, r.v. vec
\newcommand{\urvy}{\underline{\mathsf{y}}}    % y, r.v. vec
\newcommand{\urvz}{\underline{\mathsf{z}}}    % z, r.v. vec
\newcommand{\defas}{\triangleq} %\coloneqq
\newcommand{\reals}{\mathbb{R}}
%\newcommand{\T}{\mathrm{T}}    % transpose

% \newcommand{\E}[1]{\mathbb{E}\left[{#1}\right]}
% \newcommand{\Prob}[1]{\mathbb{P}\left({#1}\right)}
\DeclareMathOperator*{\argmax}{arg\,max}
\DeclareMathOperator*{\argmin}{arg\,min}
\DeclareMathOperator*{\argsup}{arg\,sup}
\DeclareMathOperator*{\arginf}{arg\,inf}
\DeclareMathOperator{\Var}{Var}
\DeclareMathOperator{\Cov}{Cov}
\DeclareMathOperator{\MSE}{MSE}
\DeclareMathOperator{\In}{\mathbb{I}}
\DeclareMathOperator{\E}{\mathbb{E}}
\DeclareMathOperator{\Prob}{\mathbb{P}}
\DeclareMathOperator{\sign}{sign}
\DeclareMathOperator{\st}{s.t.}
\newcommand\independent{\protect\mathpalette{\protect\independenT}{\perp}}
\def\independenT#1#2{\mathrel{\rlap{$#1#2$}\mkern2mu{#1#2}}}
\usepackage{algorithmic}
\usepackage{algorithm}

\newtheorem{Def}{Definition}
\begin{document}
\courseheader

\newcounter{hwcnt}
\setcounter{hwcnt}{14} % set to the times of Homework

\begin{center}
  \underline{第\thehwcnt 周作业} \\
\end{center}
\begin{flushleft}
  赵丰\quad \studentID\hfill
  \today
\end{flushleft}
\hrule

\vspace{2em}

\flushleft
\rule{\textwidth}{1pt}
\begin{itemize}
\item {\bf Acknowledgments: \/} 
  This coursework refers to textbook.  
\item {\bf Collaborators: \/}
  \begin{itemize}
  \item 9 was solved with the help from solution manual of the textbook.    
  \item 18 was solved with the help from solution manual of the textbook.      
  \end{itemize}
\end{itemize}
\rule{\textwidth}{1pt}

\vspace{2em}

I use \texttt{enumerate} to generate answers for each question:

\begin{enumerate}[label=\arabic*.]
  \setlength{\itemsep}{3\parskip}

  \item 
    我们给出一个随机算法:事件指定一运行轮数上限$K$,每轮将$G$每个顶点等概率地取三种颜色中的任一种,运行$K$轮,
    取“满足”的边数最大的顶点颜色分配方案作为算法的返回值。
    
    设$G$有$n$条边,则$n\geq c^*$,对于我们的随机选颜色,记$Z$为满足的边数,随机变量$Z_i$定义为
    \begin{equation}
        Z_i = 
        \begin{cases}
            1 & \text{第$i$条边是“满足”的}\\
            0 & \text{otherwise}
        \end{cases}    
    \end{equation}
    容易得到$E[Z_i]=\frac{2}{3}$
    则有$Z=\sum_{i=1}^n Z_i$,
    根据期望的线性性质,得到$E[Z]=\frac{2}{3}n\geq \frac{2}{3}c^*$,因此我们的算法满足期望边数不小于$\frac{2}{3}c^*$。
    
  \setcounter{enumi}{2}    
    \item \begin{enumerate}[label=(\alph*)]
    \item 
      考虑$S$中任意两个进程,若它们有冲突,则在$G$中它们是邻边,但根据进入$S$的规则,这两个进程获得的随机值
      都取1,$S$中节点(某个进程)的相邻节点(冲突进程)随机值取零矛盾。因此集合$S$是无冲突的。
      随机变量$S_i$定义为
    \begin{equation}
        S_i = 
        \begin{cases}
            1 & \text{第$i$个进程进入集合}\\
            0 & \text{otherwise}
        \end{cases}    
    \end{equation}
    容易得到$E[S_i]=\frac{1}{2^{d+1}}$
    则有$S=\sum_{i=1}^n S_i$,
    根据期望的线性性质,得到$E[S]=\frac{n}{2^{d+1}}$
    
    \item 
    \begin{equation}
        E[S]=np(1-p)^d
    \end{equation}
    $\frac{d E[S]}{d p}=0 \rightarrow p=\frac{1}{d+1}$
    \begin{equation}
        E[S]=\frac{nd^d}{(d+1)^{d+1}}
    \end{equation}
    
    
    \end{enumerate}

  \setcounter{enumi}{8}    
    \item   卖家采取的策略是首先记录前$\frac{n}{2}$个买家中报价的最高价$p$,然后
    接受后$\frac{n}{2}$个买家中第一个大于$p$的出价。
    在该策略下,卖家得到最高报价事件包含次高价在前一半买家,最高价在后一半买家。
    根据排列组后的规律,后一个事件发生的概率为$\frac{n}{4(n-1)}>\frac{1}{4}$,因此卖家
    可以用不小于$\frac{1}{4}$的概率接受$n$个报价中的最高价。
  \setcounter{enumi}{17}    
      \item \begin{enumerate}[label=(\alph*)]

    \item 最小权顶点覆盖问题是指图$G$中所含顶点的权之和最小的顶点覆盖。对于这个问题,表中的随机算法不是$c-$近似的算法,
    反例如下:考虑一个只有一条边的图,一个顶点权值为$2c$,另一个顶点权值为1。最优的最小权顶点覆盖权值之和为1,但按照表中的
    随机算法,为$c+\frac{1}{2}$,不能被最优解的$c$倍控制住,因此举的例子说明了表中的随机算法不是$c-$近似的算法。
    \item 最小规模顶点覆盖问题指图$G$中所含顶点数目最少的顶点覆盖。
    对于这个问题,我们下面证明表中的随机算法是$2-$近似的算法。
    设$p_e$表示算法中边$e$作为未覆盖的边被选中的概率,$S$是算法返回的顶点覆盖的集合,$\delta(v)$表示与顶点$v$关联的边。
    算法给出的顶点覆盖的期望$|S|$即为选中的边的数目,即有
        \begin{equation}
           S=\sum_{e\in E} p_e
        \end{equation}
        记$|\delta(v)|=v_m$,对于任意$e_i\in \delta(v)$,如果$e_i$被选中了,那么对于其他的边$e_j \in \delta(v)$,最多只有$\frac{1}{2}$
        的概率被选中(当顶点$v$没被加到$S$中时)。因此不等式$p_{e_i}\leq \frac{1}{2} p_{e_j}$成立。
        所 以
    \begin{equation}
    \sum_{e\in \delta(v)} p_e\leq p_{e_j}\sum_{i\geq 0} \frac{1}{2^i}\leq 2
    \end{equation}
    设$S^*$为最优解对应的顶点覆盖的集合,
        \begin{equation}
           |S| \leq \sum_{v\in S^*} \sum_{e\in \delta(v)} p_e \leq \sum_{v\in S^*} 2 = 2|S^*|
        \end{equation}
    因此    表中的随机算法是$2-$近似的算法。
        \end{enumerate}
  \end{enumerate}

\end{document}
%%% Local Variables:
%%% mode: latex
%%% TeX-master: t
%%% End:
    \begin{equation}
    \end{equation}
