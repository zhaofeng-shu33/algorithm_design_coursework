\documentclass{article}
\usepackage{xeCJK}
\setCJKmainfont[AutoFakeBold]{SimSun}
\usepackage{amsmath,esint,amssymb,amsthm}
\usepackage{algorithmic}
\usepackage{algorithm}
\usepackage{mathtools}
\newtheorem{remark}{remark}
\DeclarePairedDelimiter{\floor}{\lfloor}{\rfloor}
\DeclareMathOperator{\sgn}{sgn}
\begin{document}
\title{第七周作业}
\author{赵丰,2017310711}
\maketitle

\textbf{Ch5.3}

\begin{remark}
每次扔掉两张(不一样),扔一张(一样),保证过半的一类仍是过半。
当只剩下一张或两张,再与扔掉中的所有卡片比较一遍,检查一遍。算法时间复杂度为$O(n)$
\end{remark}

我们将要说明对于规模为$O(n)$的问题,存在算法$\mathcal{A}$的
时间复杂度为$T(n)$,满足$T(n)=2T(\frac{n}{2})+f(n)$,且$f(n)=O(n)$。
则由Master Theorem, $\mathcal{A}$ 的时间复杂度为 $O(n\log n)$

$\mathcal{A}$的非形式化描述为:

Divide: 将$n$张银行卡等分为2份

Conquer: 对每份(约有$\frac{n}{2}$)张银行卡运行判断算法,拿到2个布尔值,指示在每份中是否有一半以上的卡相互等价。

Combine: 若2个布尔值均为 false, 则算法返回 false; 否则对每一个返回 true 的份,取其中
一半以上的相互等价的卡中的一张(算法的返回值之一),与另一份的每一张逐份比较,若在另一份中匹配到超过一半以上的卡,则算法返回 true,
若处理完每一个返回 true 的份,算法返回 false。

初始:若该份只有1张卡,返回 true和该张卡; 若有2张卡,输入到等价性测试器中返回测试器的结果,且若两张卡等价,再返回其中一张卡。

对算法$\mathcal{A}$,实际实现中,可以使用固定长度的数组存储$n$张银行卡,每次递归调用时只需给出子问题数组的头指针和尾指针。
注意在 Combine 的过程中,2个布尔值均为 false 说明2个子问题中均没有超过一半的卡相互等价,可用反证法说明当前问题不可能有超过
一半的卡相互等价。 当2个布尔值均为 true 时, Combine 过程可能达到最大的时间复杂度,即有可能比较 $2*O(\frac{n}{2})$次,因此
$f(n)=O(n)$,即我们用分治的思想给出了一个时间复杂度为 $O(n\log n)$的有效算法。

\textbf{Ch5.4}
为计算$n$个$\frac{F_j}{q_j}$,定义函数:
$f(i)=C q_i$,
\begin{equation}
g(i)=\begin{cases}
\frac{\sgn(i)}{i^2} & i\neq 0\\
0 & i=0 \\
\end{cases}
\end{equation}
其中$\sgn(i)$为符号函数。
则
\begin{align}
(f * g)(j)= &\sum_{i} Cq_i g(j-i)\\
= &\sum_{i<j} \frac{C q_i}{(j-i)^2}-\sum_{i>j}\frac{C q_i}{(j-i)^2}
\end{align}
即$f$与$g$的卷积即为函数$F$,
实际计算中$f$只能在$\{1,2,\dots,n\}$处取值,而$g$只能在$\{-n,-n+1,\dots,n-1,n\}$处取值。因此可通过
序列$[f(1),\dots,f(n)]$与序列$[g(-n),\dots,g(n)]$作卷积,得到长为$3n$的序列。
这长为$3n$的序列中间的$n$个值乘以相应的$q_j$后即为对应的
$F_1,F_2,\dots,F_j$。

于是可利用FFT以$O(n\log n)$的时间复杂度计算出各点的$F_j$。

\textbf{Ch5.5}

我们给出的算法的非形式化描述为:

Divide: 将$n$条直线等分为2份

Conquer: 对每份(约有$\frac{n}{2}$)运行算法,返回该份所有的交点从小到大排序的数组$L$,
以及每个区间上最高的直线序号与其直线方程(可以实现为另一个数组$S$,长度比$L$多1)。

Combine: 基本思路是对于两份交点表$L_1,L_2$和直线序号表$S_1,S_2$,由于$L_1,L_2$中的交点均排好了序,可以在$O(n)$的时间对两分交点表
做归并,得到总的排序表,并且每在总的排序表$L$加入一个元素,即考虑其作为区间右端点的最小区间上最高的直线
(考虑的区间$I$为$(a_{i-1},a_i)$,其中$a_i$为$L$的第$i$个元素)
取好$I$后,最高的直线只需分别从$S_1,S_2$中对应位置比较后取出即可。
%为此,我们分别针对每个$S_k$维护一个指针,指针初始化指向各$S_k$的
%第一个元素,当$S_k$对应的$L_k$有交点被加入$L$中时,在做完当前比较后即把$S_k$的指针加1。

初始:若该份只有1条直线,返回交点集为空,区间只有实轴,其上直线序号为唯一1条直线的序号;
%若该份有2条直线$l_i,l_j$,若两直线不平行,
%返回交点集只有两直线交点$a$,$(-\infty, a)$上较高的直线为$l_i$,$(a,\infty)$上较高的直线为$l_j$。
%若平行,返回交点集为空,区间只有实轴,直线序号为截距较大的那条直线的序号。

我们将上面Combine 的步骤以伪码的形式给出(见下页)。
\begin{algorithm}
\caption{Combine步骤}
\begin{algorithmic}[1]
\STATE 初始化$L,S$为空表
\STATE 初始化$S_1$指针$p_1=1$,$S_2$指针为$p_2=1$,$L_1$指针$l_1=1$,$S_2$指针为$l_2=1$
\WHILE{$l_1$<=$L_1$.length 或 $l_2$<= $L_2$.length}
\IF{$L_2$为空$\,\,$ 或 $L_1(l_1)<L_2(l_2)$}
\STATE 将$L_1(l_1)$加入$L$
\STATE $l_1\leftarrow l_1+1$ 
\ELSE
\STATE 将$L_2(l_2)$加入$L$
\STATE $l_2\leftarrow l_2+1$
\ENDIF
\IF{$L$只有一个元素}
\STATE $I$=($-\infty$,$L_1$(1)]
\ELSIF{$l_1$>$L_1$.length 且 $l_2$> $L_2$.length}
\STATE $I$=[$L$.last,$+\infty$)
\ELSE
\STATE $I$=[$L$.last.previous,$L$.last]
\ENDIF
\IF{$S_1(p_1)$与$S_2(p_2)$在$I$上没有交点}
    \IF{$S_1(p_1)>S_2(p_2)$}
    \STATE 将 $S_1(p_1)$ 加入 $S$
    \STATE $p_1 \leftarrow p_1+1$
    \ELSE
    \STATE 将 $S_2(p_2)$ 加入 $S$
    \STATE $p_2 \leftarrow p_2+1$
    \ENDIF
\ELSE
\STATE 求出$S_1(p_1)$与$S_2(p_2)$在$I$上的交点$a$
\STATE 将$a$插入到$L$中倒数第二个位置
\STATE 将 $S_1(p_1),S_2(p_2)$按斜率从小到大依次加入$S$
\STATE $p_1,p_2$均加1
\ENDIF
\ENDWHILE
\RETURN $L,S$
\end{algorithmic}
\end{algorithm}

该算法Combine 步骤 While 循环只有n步,单步循环可用$O(1)$时间完成,所以总的时间复杂度为$O(n\log n)$,
并且上述算法返回的$S$表即为所有可见的直线。

\textbf{Ch5.6}
我们从二叉树的根节点出发,初始时探查根节点的$x_v$值,当前节点为根节点,
每次探查当前节点的两个子节点的$x_v$,若子节点的$x_v$均比当前节点大,则当前节点即为局部最小节点;
否则取子节点中$x_v$较小的那个置为当前节点,若当前节点没有子节点,则当前节点即为局部最小节点。

该算法至多探查$2d-1$个节点,其中$d=\log_2 (n+1)$为二叉树的深度,因此该算法的时间复杂度为$O(\log n)$。


\end{document}
