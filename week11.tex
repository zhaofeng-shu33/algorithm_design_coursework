% Homework template for Inference and Information
% UPDATE: September 26, 2017 by Xiangxiang
\documentclass[a4paper]{article}
\usepackage{amsmath, amssymb, amsthm}
% amsmath: equation*, amssymb: mathbb, amsthm: proof
\usepackage{moreenum}
\usepackage{mathtools}
\usepackage{url}
\usepackage{enumitem}
\usepackage{graphicx}
\usepackage{subcaption}
\usepackage{booktabs} % toprule
\usepackage[mathcal]{eucal}
%% Definitions for Inference and Information
%% UPDATE: September 26, 2017 by Xiangxiang 
\newcommand{\theterm}{Fall 2017}
\usepackage{xeCJK}

\setCJKmainfont[AutoFakeBold]{SimSun}

\newcommand{\thecoursename}{
Tsinghua University\\
%\vspace*{0.1in}
算法设计与分析
}
\newcommand{\studentID}{2017310711}
\newcommand{\courseheader}{
\vspace*{-1in}
\begin{center}
\thecoursename \\
\theterm
\vspace*{0.1in}
\hrule
\end{center}
}
\newcommand{\NP}{$\mathcal{NP}$}
\newcommand{\rvx}{\mathsf{x}}    % x, r.v.
\newcommand{\rvy}{\mathsf{y}}    % y, r.v.
\newcommand{\rvz}{\mathsf{z}}    % z, r.v.
\newcommand{\urvx}{\underline{\mathsf{x}}}    % x, r.v. vec
\newcommand{\urvy}{\underline{\mathsf{y}}}    % y, r.v. vec
\newcommand{\urvz}{\underline{\mathsf{z}}}    % z, r.v. vec
\newcommand{\defas}{\triangleq} %\coloneqq
\newcommand{\reals}{\mathbb{R}}
%\newcommand{\T}{\mathrm{T}}    % transpose

% \newcommand{\E}[1]{\mathbb{E}\left[{#1}\right]}
% \newcommand{\Prob}[1]{\mathbb{P}\left({#1}\right)}
\DeclareMathOperator*{\argmax}{arg\,max}
\DeclareMathOperator*{\argmin}{arg\,min}
\DeclareMathOperator*{\argsup}{arg\,sup}
\DeclareMathOperator*{\arginf}{arg\,inf}
\DeclareMathOperator{\Var}{Var}
\DeclareMathOperator{\Cov}{Cov}
\DeclareMathOperator{\MSE}{MSE}
\DeclareMathOperator{\In}{\mathbb{I}}
\DeclareMathOperator{\E}{\mathbb{E}}
\DeclareMathOperator{\Prob}{\mathbb{P}}
\DeclareMathOperator{\sign}{sign}
\DeclareMathOperator{\st}{s.t.}
\newcommand\independent{\protect\mathpalette{\protect\independenT}{\perp}}
\def\independenT#1#2{\mathrel{\rlap{$#1#2$}\mkern2mu{#1#2}}}
\usepackage{algorithmic}
\usepackage{algorithm}

\begin{document}
\courseheader

\newcounter{hwcnt}
\setcounter{hwcnt}{11} % set to the times of Homework

\begin{center}
  \underline{第\thehwcnt 周作业} \\
\end{center}
\begin{flushleft}
  赵丰\quad \studentID\hfill
  \today
\end{flushleft}
\hrule

\vspace{2em}

\flushleft
\rule{\textwidth}{1pt}
\begin{itemize}
\item {\bf Acknowledgments: \/} 
  This coursework refers to textbook.  
\item {\bf Collaborators: \/}
  \begin{itemize}
  \item 6 was solved with the help from solution manual of the textbook.  
  \item 7 was solved with the help from solution manual of the textbook.    
  \end{itemize}
\end{itemize}
\rule{\textwidth}{1pt}

\vspace{2em}

I use \texttt{enumerate} to generate answers for each question:

\begin{enumerate}[label=\arabic*.]
  \setlength{\itemsep}{3\parskip}
  \setcounter{enumi}{4}
  \item 
     记平均负载为$a=\frac{1}{10}\sum_{k}t_k\geq 300$
     在负载均衡问题的证明中,结合负载的特性有$t_j\leq 50$,$L_i-t_j\leq \frac{1}{10}\sum_{k}t_k=a$。
     其中$L_i$是用时最长的机器(编号为$i$)的时长,$t_j$是最后放到第$i$台机器上的job。因此$L_i\leq a+50\leq a+ \frac{a}{6}\leq 1.2a$。
     因此工期总是不超过平均负载的20\%。
     
  
    \begin{algorithm}
    \caption{针对两类机器的负载均衡问题的算法}\label{alg}
    \begin{algorithmic}[1]
    \FOR{$i=1$ to $m+k$}
    \STATE $L_i \leftarrow 0$
    \STATE $J(i)\leftarrow \emptyset$
    \ENDFOR
    \FOR{$j=1$ to $n$}
    \STATE $i=\argmin_k L_k$
    \STATE $J(i)\leftarrow J(i)\cup j$
    \IF{$j$是快速机器}
    \STATE $L_i\leftarrow L_i+\frac{t_j}{2}$
    \ELSE
    \STATE $L_i\leftarrow L_i+t_j$    
    \ENDIF
    \ENDFOR
    \RETURN $J(1),\dots,J(m+k)$        
    \end{algorithmic}
    \end{algorithm}  
    \item 
    算法如表一所示,是多项式时间的。下面我们论证算法一返回的工期即$L\triangleq\max_i L_i \leq 3L^*$,其中$L^*$表示最优工期。
    
    首先有
    \begin{equation}\label{eq:1}
        L^*\geq \frac{t_i}{2},\forall i
    \end{equation}    
    考虑最优作业调度方案:对快速机器,假设集合$S_i$表示分配给第$i$台机器的作业,则$\sum_{j\in S_i} t_j \leq 2L^*$,
    对$k$台机器对应的不等式求和得$\sum_{j\in S} t_j \leq 2kL^*$,其中$S$表示分配给快速机器的所有作业。
    同理可得,对慢速机器,$\sum_{j\notin S} t_j \leq mL^*$,两式相加得
    \begin{equation}\label{eq:2}
        L^* \geq  \frac{\sum_i t_i}{m+2k}   
    \end{equation}
    假设算法一的作业调度方案的最长工时$L$是在快速机器上取到,其最后一件作业时长为$\frac{t_j}{2}$
    则对于所有的快速机器$i$,$L-\frac{t_j}{2}\leq \sum_{k \in S_i} \frac{t_k}{2}$
    对于所有的慢速机器$i$, $L-\frac{t_j}{2}\leq \sum_{k \not\in S_i} t_k$,
    求和并结合(\ref{eq:1},\ref{eq:2})有$L\leq  \frac{t_j}{2}+ \frac{\sum_i t_i}{m+2k} \leq 2L^*$
    
    假设算法一的作业调度方案的最长工时$L$是在慢速机器上取到,其最后一件作业时长为$t_j$,同理得
    $L\leq  t_j +\frac{\sum_i t_i}{m+2k} \leq 3L^*$

    综上,算法一返回的工期不超过最优工期的3倍。

    \begin{algorithm}
    \caption{击中集问题的近似算法}\label{alg2}
    \begin{algorithmic}[1]
    \STATE 初始化n乘m的布尔矩阵A使得$A(i,j)$表示$a_i$是否在$B_j$中出现
    \FOR{$i=1$ to $m$}
    \STATE $p_i \leftarrow 0$
    \ENDFOR
    \WHILE{矩阵$A$中存在某一列$j$使得$w_i>\sum_{k=1}^m A(i,k)p_k,\forall i\in S=\{i|A(i,j)=1\}$}
    \STATE $p_i\leftarrow p_i+\min_{k\in S}(w_k-\sum_{j=1}^m A(k,j)p_k)$
    \ENDWHILE
    \RETURN $H=\{a_i|w_i=\sum_{j=1}^m A(i,j)p_j\}$
    \end{algorithmic}
    \end{algorithm}  
    
    \setcounter{enumi}{3}
    \item 
    表二给出了击中集问题的近似算法,返回的集合为击中集$H$。首先注意到该算法对每个$p_j$只能修改一次,
    且一旦修改后最终返回的$H$与$B_j$交为非空。
    因此假如$B_j\cap H=\emptyset$,则$p_j$没有被修改过,仍保持初始值0。则while循环的条件得到满足,
    这与算法提前返回$H$矛盾。
    当$|B_i|=2,\forall i$时,即为带权的顶点覆盖问题,
    此时$b=2$,即可以在多项式时间内找到总权不超过最优覆盖2倍的顶点覆盖。
    因此表二中的近似算法是带权的顶点覆盖的近似算法的推广,其分析过程也与带权的顶点覆盖算法分析
    类似。首先我们看到,在算法运行过程中始终有下面的关系成立。
    \begin{equation}
        w_i \geq \sum_{j=1}^m A(i,j)p_j,\forall 1\leq i\leq n    
    \end{equation}
     设$H$是任一击中集,将上面若干个式子相加有
     \begin{align}\label{eq:pjwH}
     \sum_{j=1}^m p_j \leq \sum_{a_i \in H}(\sum_{j=1}^m A(i,j)p_j)\leq \sum_{a_i\in H} w_i:=w(H)
     \end{align}
     假设$H^*$是使得元素总权最小的击中集。我们证明$w(H)\leq b\times w(H^*)$,
     其中$H$是表二中的近似算法返回的击中集,b表示矩阵$A$各列和的最大值。
     \begin{align}
     w(H)=&\sum_{a_i\in H} w_i=\sum_{a_i\in H} \sum_{j=1}^m A(i,j)p_j\\
     \leq &\sum_{i=1}^n \sum_{j=1}^m A(i,j)p_j\leq b\sum_{j=1}^m p_j\\
     \leq & b\times w(H^*),\text{by \eqref{eq:pjwH}}
     \end{align}
     因此表二给出的算法可以找到一个总权不超过最小可能b倍的击中集。
     
    \begin{algorithm}
    \caption{最大宽度的近似算法}\label{alg3}
    \begin{algorithmic}[1]
    \FOR{$i=1$ to $m$}
    \STATE $L_i \leftarrow 0$
    \STATE $A(i)\leftarrow \emptyset$
    \ENDFOR
    \FOR{$j=1$ to $n$}
    \STATE $i=\argmin_k L_k$
    \STATE $A(i)\leftarrow A(i)\cup j$
    \STATE $L_i\leftarrow L_i+v_j$    
    \ENDFOR
    \RETURN $\min_{j} L(j)$        
    \end{algorithmic}
    \end{algorithm}  

    \setcounter{enumi}{6}
    \item 
    \begin{enumerate}[label=(\alph*)]
    \item
    本问题的算法和负载均衡问题一致,即在无需按顾客的权值排序的情况下进行离线处理,表三给出了求解问题使用的近似算法,
    其中$v_j$表示第$j$位顾客的权值。
    设$L^*$为给定$v_j$后所能达到的最大宽度,根据问题的性质,最大宽度不超过顾客权值关于广告数量的平均,即
    \begin{equation}\label{eq:LmSV}
        L^* \leq \frac{1}{m} \sum_{j=1}^n v_j:=\frac{1}{m} S_v
    \end{equation}
    另一方面,根据本问中额外的假设$v_j\leq \frac{1}{2m}S_v,\forall j$,我们断言表三给出的算法返回的宽度$L$满足如
    下的不等式:
    \begin{equation}\label{eq:12mSv}
        \frac{1}{2m} S_v \leq L
    \end{equation}
    反设$L<\frac{1}{2m} S_v$,设表三给出的算法最小值在$j$处达到(即$L_j=L$),
    在处理最后一位顾客$v_n$(没有考虑广告$A(j)$)之前,由贪心算法的性质,各广告的权值满足$\tilde{L_i}\leq L,\forall i$
    即$L_i\leq L+v_n$,对$i$求和得
    \begin{equation}
        \sum_{i=1}^m L_i \leq mL+m\times v_n
    \end{equation}
    因为$\sum_{i=1}^m L_i=\sum_{i=1}^n v_i=S_v$,根据题目中额外的假设,上式右边严格小于$\frac{1}{2}S_v+\frac{1}{2}S_v$,
    所以我们得到矛盾。因此根据\eqref{eq:LmSV}式和\eqref{eq:12mSv}式,我们直接推出:
    \begin{equation}
        \frac{L^*}{2}\leq L
    \end{equation}
    因此我们给出了一个近似宽度到因子2范围内的多项式时间算法。
    
    \item
        考虑6位顾客,2个广告。6位顾客的权值为$\{1,3,2,1,2,3\}$,$S_v=12$,满足所有顾客权值不超过$S_V/(2\times 2)$。
        最优解为6,但运行表三的算法,返回的最大宽度为5,因此表三算法没有找到最优解。
    \end{enumerate}
    
     \end{enumerate}

\end{document}
%%% Local Variables:
%%% mode: latex
%%% TeX-master: t
%%% End:
     \begin{equation}
    \end{equation}
